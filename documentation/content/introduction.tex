\chapter{Introduction}

\section{Motivation}
A family member of mine approached me with the request of looking into ways to monitor his beehives. He wanted to collect some basic information about the colony like weight and temperature. This information should be made available over the internet and should be displayed in an easy-to-read format. The goal was to have an overview about the colony and to be able to react to changes in the colony.

After listening to the research of my family member, I decided to look at the solutions that are commercially available. These are quite expensive, proprietary and not open source. Most of them also required a subscription from the manufacturer to work. I decided to look into the possibility of building my own solution. This would allow me to have a solution that is tailored to my needs and is open source.

This project was realized in the context of my Project 2 course at the Berne University of Applied Sciences.

\pagebreak
\section{Requirements}
To start of this project I had to define a set of requirements that the solution should fulfill. These requirements were then used to guide the design of the solution. I sat down with my family member and discussed the functional and non-functional requirements. The functional requirements were the features that the solution should have. The non-functional requirements were the requirements that were not directly related to the functionality of the solution.

\subsection{Functional Requirements}
\begin{itemize}
    \item The solution should be able to be deployed in any location.
    \item The solution should include its own power supply.
    \item The solution should be able to connect to the internet in a location independent manner.
    \item The solution should be able to measure the weight of the beehive.
    \item The solution should be able to measure the temperature of the beehive.
    \item The solution should publish the data it collects to the internet.
    \item The solution should save the data it collects in a persistent manner.
    \item The solution should not require any special equipment to be installed.
    \item The solution should be able to be installed in a standard beehive.
    \item The solution should be implementation independent. Meaning that the individual components can be replaced with other components that fulfill the same function.
\end{itemize}
\subsection{Non-Functional Requirements}
\begin{itemize}
    \item The solution should be as energy efficient as possible.
    \item The solution should be able to generate its own power.
    \item The solution should be resistant to the elements.
    \item The solution should be as cost-effective as possible.
    \item The solution should be simple to build.
    \item The solution should be easy to maintain.
    \item The solution should be easy to extend.
\end{itemize}

\section{Method}
Given that I was the only one working on this project, I decided to use a simple agile development process. This process did not follow any classical agile methodology, but rather a simplified version of Kanban. The process consisted of me defining a set of tasks that needed to be done. These tasks were then put into a backlog. I then picked a task from the backlog and started working on it. Once the task was completed, I moved onto the next one. This process was repeated until the project was completed.
The tasks were grouped into a set of categories.

\subsection{Phases}

\subsubsection{Planning}
\textbf{Task: Research existing solutions}
\begin{itemize}
    \item Research existing solutions.
    \item Compare existing solutions.
    \item Analyze features of existing solutions.
\end{itemize}

\textbf{Task: Define requirements}
\begin{itemize}
    \item Define functional requirements.
    \item Define non-functional requirements.
    \item Discuss requirements with family member.
\end{itemize}

\textbf{Task: Rough Hardware Planning}
\begin{itemize}
    \item Define overall setup.
    \item Define components.
\end{itemize}
\subsubsection{Building}

\subsubsection{Software}

\subsubsection{Documentation}