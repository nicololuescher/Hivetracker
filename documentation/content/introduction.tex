\chapter{Introduction}

\section{Motivation}
A family member of mine approached me with the request of looking into ways to monitor their beehives. They wanted to collect some basic information about their colonies like weight and temperature. This information should be made available over the internet and should be displayed in an easy-to-read format. The goal was to have an overview over the colonies and to be able to react to changes.

After listening to the request of my family member, I decided to look at the solutions that are commercially available. These are quite expensive, proprietary and not open source. Most of them also required a subscription from the manufacturer to work. I decided to look into the possibility of building my own solution. This would allow me to have a solution that is tailored to my needs and is open source.

This project was realized in the context of my Project 2 course at the Berne University of Applied Sciences.

\newpage
\section{Method}
Given that I was the only one working on this project, I decided to use a simple \gls{agile} development process. This process did not follow any classical agile methodology, but rather a simplified version of \gls{Kanban}. The process consisted of me defining a set of tasks that needed to be done. These tasks were then put into a \gls{backlog}. I then picked a task from the backlog and started working on it. Once the task was completed, I moved onto the next one. This process was repeated until the project was completed.
The tasks were grouped into a set of categories.

\subsection{Phases}

\subsubsection{Planning}
\textbf{Task: Research existing solutions}
\begin{itemize}
    \item Research existing solutions.
    \item Compare existing solutions.
    \item Analyze features of existing solutions.
\end{itemize}

\textbf{Task: Define requirements}
\begin{itemize}
    \item Define functional requirements.
    \item Define non-functional requirements.
    \item Discuss requirements with family member.
\end{itemize}

\textbf{Task: Rough Hardware Planning}
\begin{itemize}
    \item Define overall setup in regard to the functional requirements.
    \item Define components.
    \item Define communication protocols.
\end{itemize}

\textbf{Task: Specific Hardware Planning}
\begin{itemize}
    \item Define specific components.
    \item Define physical setup.
    \item Define wiring.
\end{itemize}

\newpage
\textbf{Task: Software Planning}
\begin{itemize}
    \item Define software stack.
    \item Define software architecture.
    \item Define data format.
    \item Define communication protocols.
    \item Define data flow.
    \item Define user interface.
\end{itemize}

\subsubsection{Implementation}
\textbf{Task: Build Prototype}
\begin{itemize}
    \item Procure components.
    \item Cut raw materials.
    \item Assemble prototype.
    \item Solder components.
    \item Test Components.
\end{itemize}

\textbf{Task: Server Setup}
\begin{itemize}
    \item Setup Docker environment.
    \item Setup MQTT broker.
    \item Setup PostgreSQL database.
    \item Setup NodeJS server.
    \item Integrate Express into NodeJS server.
    \item Integrate SequelizeJS into NodeJS server.
    \item Create database models.
    \item Persist MQTT messages into database.
    \item Create API routing.
    \item Create API endpoints.
\end{itemize}

\newpage
\textbf{Task: User Interface Setup}
\begin{itemize}
    \item Setup Angular environment.
    \item Integrate Tailwind CSS into Angular.
    \item Integrate ChartJS into Angular.
    \item Create basic layout.
    \item Create data service.
    \item Visualize data with ChartJS.
\end{itemize}

\subsubsection{Testing}
\textbf{Task: Test Setup}
\begin{itemize}
    \item Setup test environment.
    \item Test weight sensor.
    \item Test weight calibration.
    \item Test temperature sensor.
    \item Test power consumption.
    \item Test user interface.
    \item Test end to end communication.
\end{itemize}

\subsubsection{Documentation}
\textbf{Task: Documentation}
\begin{itemize}
    \item Setup LaTeX environment.
    \item Write documentation.
\end{itemize}